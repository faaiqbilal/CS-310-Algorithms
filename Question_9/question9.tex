\documentclass[]{article}
% \usepackage{graphicx}

\begin{document}

\begin{center}
    \vspace*{1cm}

    \textbf{Question 9.}

    \vspace{0.5cm}
     Algorithms Assignment 1

    \vspace{0.15cm}

    \textbf{Faaiq Bilal} \\ 
    \textbf{23100104}
         
\end{center}

\section{Part 1}
We can divide the algorithm into each of its steps, analyze their running time complexity to get bounds for our algorithm's complexity. \\ \\
The outermost \emph{for} loop will run over all the elements of the array of size $n$. This step will have a complexity of $O(n)$. The following functions are nested inside this \emph{for} loop, which means that they will be repeated $n$ times as well \\ \\
The nested \emph{for} loop will run for a maximum of $n-2$ times, which is still bounded by $O(n)$. As noted above, this \emph{for} loop will be repeated $O(n)$ times as well. \\ \\
As the final piece of this equation, the integers inside the Array $A$ have to be summed up. The maximum number of computations for a specific $B[ij]$ is equal to the largest difference between $j$ and $i$. This is bounded by $O(n)$, as $i$ ranges $1 \rightarrow n$ and $j$ ranges $i+1 \rightarrow n$.\\ \\
It should be noted that this summation is nested inside the above for loop, hence it will be repeated $O(n^2)$ times. \\ \\
As each computation is of order $O(n)$, and is repeated $O(n^2)$ times, the overall computational complexity of the algorithm will have an upper bound of $O(n^3)$. 

\section{Part 2}
We can use the breakdown from Part 1 to help make a point here as well. \\ \\
The outermost for loop will run $n$ times, the inner for loop will run from $1 \rightarrow n-2$ times as well. \\
The innermost summation runs 



\end{document}