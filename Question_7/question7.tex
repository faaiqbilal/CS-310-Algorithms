\documentclass[]{article}
% \usepackage{graphicx}

\begin{document}

\begin{center}
    \vspace*{1cm}

    \textbf{Question 7.}

    \vspace{0.5cm}
     Algorithms Assignment 1

    \vspace{0.15cm}

    \textbf{Faaiq Bilal} \\ 
    \textbf{23100104}
         
\end{center}

\section{Listing from smallest to largest complexity}

$2 \log_{10} 10^{200}$, $10^{200}$ Constant time ($O(1)$) \\ 
They belong to the same class as they have a constant size. The complexity does not depend on the value of n. \\ \\
$n^2 + \sqrt{n}$, $n^2 + 7 \log n^2$, $ n^2 \log 300  $  Polynomial (Quadratic, $O(n^2)$) time \\ 
These expressions all belong to the same complexity class as they are bound by $n^2$. We know this because $n^2$ is the largest term in each of the expressions, and is hence the largest bound. \\ \\
$ n^2 \log n $ Polynomial time ($O(n^2 \log n)$) \\ 
This is not as simple as putting it in under $O(n^2)$, as the $log(n)$ term is multiplying, and will hence always overtake $c \cdot n^2$ as n grows infinitely large. The bound for this can also be given as $O(n^3)$, however, the immediate tightest bound is still in the form $O(n^2 \log n)$ \\ \\g
$ n^4 $, $ n^4 + n^2 + n^2 \log n $ Polynomial time (Quartic, $O(n^4)$) \\ 
All of the expressions belonging to this class have $n^4$ as the largest term, and hence $n^4$ bounds the equation. Hence, we can say that they belong to the same complexity class. \\ \\
$ 3^{n^2} $ Exponential time ($k^{f(n)}$) \\ \\


\end{document}