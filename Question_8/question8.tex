\documentclass[]{article}
% \usepackage{graphicx}

\begin{document}

\begin{center}
    \vspace*{1cm}

    \textbf{Question 8.}

    \vspace{0.5cm}
     Algorithms Assignment 1

    \vspace{0.15cm}

    \textbf{Faaiq Bilal} \\ 
    \textbf{23100104}
         
\end{center}

\section{Part 1}
This can be disproved with a counter-example. If we take $ f(n) = n$ and $ g(n) = n^2$ \\ 
Then the first condition is satisfied as $ n < c \cdot n^2 $ for all $n > 0 $ and $c \geq 1$  
\\ However, the converse is not necessarily true, as $n^2$ grows faster than $c \cdot n $.
\\
Hence, in this case, $g(n) \not\in f(n) $

\section{Part 2}
The proof of this statement can be expressed in two parts:
\subsection{Proof part 1, Big O}
For this, we can start off by supposing that $f(n) > g(n)$. This supposition doesn't actually matter, it only helps us in denoting one function as being larger than the other. The roles could very easily have been swapped. 
\\ \\
In such a case, the following must also be true for any constant $c_0 \geq 2$ : \\
$ c \cdot f(n) \geq f(n) + g(n) $ 
\\ \\
This proves that $f(n) + g(n) $ $ \epsilon $ $ max(f(n), g(n)) $

\subsection{Proof part 2, Big Omega}
This part is much simpler to prove, as function $f(n)$ and $g(n)$ grow with increasing n.
\\
This would mean that $f(n) + g(n)$ must be larger than both $f(n)$ and $g(n)$ individually. It does not matter which of them is larger.
\\
Hence, we can say that $f(n) + g(n) \geq max(f(n),g(n))$
\\
$f(n) + g(n) $ $ \epsilon$ $ max(f(n), g(n)) $ \\
$f(n) + g(n) $ $ \epsilon$ $ \Omega (max(f(n), g(n)))  $
\\

\subsection{Conclusion}
As both statements above are true, then $ f(n) + g(n) $ $\epsilon $ $ \theta (max(f(n) + g(n)))$


\section{Part 3}
This statement is true as shown below: \\ \\
$f(n) = O(g(n))$ can also be written as: $f(n) \leq M \cdot g(n)$, where M is some constant $> 0$. \\
This statement can be extended as follows:
$lg(f(n)) \leq lg(M \cdot g(n))$ \\
Which will further mean that: \\
$ lg (f(n)) \leq K \cdot (lg M + lg(g(n)))  $  where K is some constant $>$ 0 \\




\end{document}